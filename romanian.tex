% !TeX spellcheck = ro_RO

\renewcommand{\listoflistingscaption}{Lista fragmente de coduri}
\renewcommand{\listingscaption}{Fragment de cod}

\newpage
\pagestyle{empty}
\selectlanguage{romanian}

	\begin{center}
		{\Large Universitatea Sapientia din Cluj-Napoca}\\\vspace{0.07in}
		{\Large Facultatea de Științe Tehnice și Umaniste, Târgu-Mureș}\\\vspace{0.07in}
		{\Large Specializarea Calculatoare}\\
		
		\vspace{2.35in}
		
		{\huge Sistem pentru Testarea Penetrabilității și}\\\vspace{0.15in}
		{\huge Descoperirea Vulnerabilităților}
		
		\vspace{0.5in}
		
		{\LARGE Proiect de Diplomă -- Extras}
		
	\end{center}
	
	\vspace{2.0in}
	
	\begin{multicols}{2}
		\begin{flushleft}
			{\Large Coordonator științific:}
		\end{flushleft}
		\columnbreak
		\begin{flushright}
			{\Large Absolvent:}
		\end{flushright}
	\end{multicols}
	\begin{multicols}{2}
		\begin{flushleft}
			{\LARGE Dr. Vajda Tamás}
		\end{flushleft}
		\columnbreak
		\begin{flushright}
			{\LARGE Bogosi Roland}
		\end{flushright}
	\end{multicols}
	
	\vspace{1.5in}
		
	\begin{center}
		{\LARGE 2016}
	\end{center}

\newpage

	\includepdf[pages=1]{declaration.pdf}

\newpage
\section*{Cuprins}

	\begingroup
	\renewcommand{\section}[2]{}
	\romaniantableofcontents
	\endgroup

\newpage
\pagestyle{fancy}
\rhead{\slshape EXTRAS}
\thispagestyle{plain}

	\begin{center}
		{\huge Sistem pentru Testarea Penetrabilității și}\\\vspace{0.1in}
		{\huge Descoperirea Vulnerabilităților}
		
		\vspace{1.5cm}
		
		{\LARGE Extras}
		
		\vspace{0.5cm}
	\end{center}

\section*{Scopul Proiectului}

	Scopul proiectului este de a dezvolta o aplicație, care poate să scaneze o rețea și să descopere vulnerabilitățile rețelei scanate.

	Aplicația trebuie să fie complet autonomă în procesul de descoperirea vulnerabilităților, fără a depinde de actualizări de la dezvoltator pentru a identifica servicii și vulnerabilități noi.

	Publicul țintă al aplicației trebuie să fie cât mai largă. Aplicația trebuie să fie un instrument util pentru cercetători în domeniul de securitate, pentru consultanți, dar și pentru administratori, chiar în cazul în care administratorii rețelei nu au experiențe semnificative în domeniul de securitate.

\subsection*{Achiziție de Date}

	Pentru a fi folositor cercetătorilor independenți în domeniul de securitate, aplicația trebuie să conțină metode de ``OSINT'' (``open-source intelligence''). Serviciile prezentate în secțiunea \ref{passive}, anume Shodan\cite{shodan16}, Censys\cite{censys15} și Mr Looquer\cite{looquer16}, sunt candidate perfecți pentru componentul al aplicației care achiziționează date prin metode pasive.
	
	Prin suportarea metodelor pentru a achiziționa date fără o scanare activă, cercetătorii în domeniul de securitate pot să lanseze o scanare pentru care vor primi rezultate instante, fără necesitatea de a avea o infrastructură largă pentru scanare.
	
	În cazul contrar, consultanții în domeniul de securitate în general trebuie să lanseze o scanare împotriva unei infrastructuri private, pentru care date nu sunt disponibile pe serviciile menționate anterior. Din acest motiv, aplicația trebuie să conțină și metode active pentru achiziționarea datelor.
	
	Indiferent de faptul că aplicația implementează mai multe moduri de a achiziționa date, ca un mijloc de redundanță, utilizatorii pot să lanseze scanarea cu o aplicație exterioară, importând rezultatele pentru procesare în final. În mod implicit, aplicația de scanare \textit{nmap} este suportat, dar orice aplicație se poate folosi care generează reporturi în format XML care sunt compatibile cu nmap; un exemplu pentru un astfel de aplicație ar fi \textit{masscan}.

\subsection*{Analizarea Datelor}

	Bannerele de serviciu (``service banner'') care au fost colectate trebuie să fie analizate într-un mod complet autonom. Contrar cu aplicațiile prezentate în secțiunea \ref{relwork}, aplicația dezvoltată nu trebuie să depindă pe extensii pentru a identifica servicii și descoperi vulnerabilități.

	Aplicația trebuie să folosească bazele de date publicate de instituția \textit{NIST}, discutat în secțiunea \ref{vulndbs}. Secțiunea \ref{matchcpe} prezintă componentele care procesează bannerele de serviciu folosind bazele de date menționate anterior, și asociază o nume \textit{CPE} într-un mod complet autonom.

	Ca un mijloc de redundanță, secțiunea \ref{patternmatch} prezintă o metodă alternativă pentru procesarea bannerelor de serviciu. Această metodă nu folosește bazele de date menționate anterior, ci o bază de date care conține nume CPE asociate cu expresii regulate, creat și menținut de dezvoltatorul aplicației sau comunitatea.

\subsection*{Sugestii de Soluționare}

	După ce serviciile scanate au fost identificate, descoperirea vulnerabilităților este o simplă operațiune de căutarea în baza de date cu vulnerabilități \textit{CVE}.
	
	Din punctul acesta, aplicația trebuie să prezinte cât mai multe informații utile utilizatorului despre serviciile și vulnerabilitățile descoperite, ca și cum ar fi punctajul de CVSS al vulnerabilităților în descoperite în infrastructură pentru analiza de risc.
	
	Ca să fie folositoare și pentru administratorii de rețea care nu au experiențe semnificative în domeniul de securitate, aplicația trebuie să încerce să genereze o listă simplă cu pașii pentru a soluționa vulnerabilitățile descoperite pe fiecare server.
	
	În cazul serverurilor unde sistemul de operare a fost identificat cu succes, aplicația poate să genereze o linie de comandă care când este rulat pe serverul cu vulnerabilități va actualiza programele vulnerabile.

\section*{Soluții Aferente}
	
\subsection*{Soluții Comerciale}
	
	Proiectul \textit{nmap} menține o listă cu instrumentele de securitate, ordonat pe baza popularității acestora\cite{sectools}. În această secțiune vor fi prezentate trei instrumente de penetrare dintre cele mai populare din lista menționată ulterior.
	
	\paragraph*{Nessus} \textit{Nessus Vulnerability Scanner}\cite{nessus} este dezvoltat și menținut de Tenable Network Security. Inițial a fost gratuit și open-source, cu funcții opționale contra cost, dar în 2005 sursa a fost închisă. SecTools.org pune Nessus pe poziția numărul unu pe popularitatea instrumentelor din această categorie. Momentan există o ediție ``Community'' care e gratuit, dar are câteva funcții limitate și se poate folosi numai pentru uz personal. Abonamentul anual pentru versiunea plătită începe de la \$2,190 per utilizator.
	
	\paragraph*{OpenVAS} \textit{Open Vulnerability Assessment System}\cite{openvas} este un fork al proiectului Nessus, din perioada în care era open-source. Momentan este dezvoltat și menținut de Greenbone Networks. SecTools.org pune OpenVAS pe poziția numărul doi pe popularitate instrumentelor din această categorie, iar popularitatea acestuia este într-o creștere continuară din cauza sursei deschise.
	
	\paragraph*{Nexpose} \textit{Nexpose Vulnerability Scanner}\cite{nexpose} este dezvoltat și menținut de Rapid7, o companie care este cunoscut în domeniul de securitate din cauza unui alt proiectul dezvoltate de ei, anume \textit{Metasploit}. SecTools.org pune Nexpose pe poziția numărul trei pe popularitate instrumentelor din această categorie. Abonamentele anuale încep de la \$2,000, dar există și o variantă ``Community'' care se poate folosii gratis pentru uz personal.
	
\subsection*{Lucrări Științifice}
	
	Anumite articole studiază eficiența al soluțiilor comerciale existente. Cele mai citate dintre aceste articole sunt \cite{holm11,bau10,doupe10}.
	
	Publicația \cite{kals06} detaliază planificarea și realizarea unei sistem de teste de penetrare pentru web, iar dintr-o altă perspectivă, articolul \cite{guo05} studiază securitatea dispozitivelor conectate la o rețea publică.
	
	\textit{ShoVAT}\cite{shovat15} prezintă o metodologie care este folosită și în aplicația dezvoltată în scopul licenței cu mici modificări, componentul fiind detaliat în secțiunea \ref{matchcpe}, vorbind despre identificarea serviciilor într-un mod autonom, folosind baza de date publicat de instituția NIST.

\section*{Realizarea}

	Aplicația realizată în cadrul tezei și scripturile asociate sunt gratuite și open source.
	
	Repertoriul git al aplicației principale se poate descărca de pe adresa \url{https://github.com/RoliSoft/Host-Scanner}, și se poate folosi sub termenii al licenței GNU General Public License version 3\cite{gplv3}.
	
	Scripturile dezvoltate pentru experimentare și prelucrarea datelor se pot descărca de pe adresa \url{https://github.com/RoliSoft/Host-Scanner-Scripts}, și se pot folosi sub termenii al licenței MIT license\cite{mit}.

\subsection*{Scanare Activă}

\subsubsection*{Scanare cu ICMP `EchoRequest'}

	Protocolul \textit{Internet Control Message Protocol} (\textit{ICMP}) definește un pachet de tip `EchoRequest', la care destinatarii vor răspunde automat cu un pachet de tip `EchoReply'. Acest pachet conține seria originală a cererii și octeți atașați, pe care inițiatorul poată să folosească să facă analize statistice.
	
	Implementația componentului folosește socluri de tip ``raw'' pentru a trimite pachete de tip `EchoRequest'. Din acest motiv, utilizatorul care lansează aplicația trebuie să aibă drepturi de administrator pe diverse sisteme de operare, de exemplu Linux, Windows, și diferite distribuții de BSD.
	
	Pachetele de acest tip nu se pot adresa la un port exact, numai la o gazdă. Din acest motiv, componentul nu poate să descopere servicii pe o gazdă, numai gazdele online într-o rețea.

\subsubsection*{Scanare cu ARP `WhoHas'}

	Mecanismul \textit{Address Resolution Protocol} (\textit{ARP}) este responsabil pentru traducerea adreselor IPv4 în adrese MAC (și vice versa) într-o rețea locală de calculatoare.
	
	Similar cu ICMP, acest component nu poate să descopere serviciile a unei gazde, numai gazdele online într-o rețea locală.

\subsubsection*{Scanare TCP}

	Scanarea porturilor TCP este realizată de componentul \mintinline{cpp}{TcpScanner}. Implementația curentă nu necesită drepturi de administrator de la utilizatorul care lansează aplicația, fiindcă aplicația folosește interfața standard de rețea al sistemului de operare pentru a conecta la destinatari.
	
	\begin{figure}[!htbp]
		\centering
		\begin{tikzpicture}
			\node at (0,0) {\Huge \faLaptop};
			\node at (4.5,0) {\Huge \faServer};
			\draw (0,-0.5) -- (0,-6.5);
			\draw (4.5,-0.5) -- (4.5,-6.5);
			\node at (-1.1,-1) {\mintinline{cpp}{connect()}};
			\draw [es](0.5,-1) -- (4,-1);
			\node at (2.25,-0.65) {SYN $i$};
			\draw [es](4,-2) -- (0.5,-2);
			\node at (5.5,-2) {\mintinline{cpp}{accept()}};
			\node at (2.25,-1.65) {SYN $j$, ACK $i+1$};
			\node at (-1,-2) {conectat};
			\draw [es](0.5,-3) -- (4,-3);
			\node at (2.25,-2.65) {ACK $j+1$};
			\draw [es](0.5,-4) -- (4,-4);
			\node at (5.5,-3) {conectat};
			\node at (2.25,-3.65) {FIN $k$};
			\node at (-0.9,-4) {\mintinline{cpp}{close()}};
			\draw [es](4,-5) -- (0.5,-5);
			\node at (2.25,-4.65) {ACK $k+1$, FIN $l$};
			\draw [es](0.5,-6) -- (4,-6);
			\node at (-0.8,-5) {închis};
			\node at (2.25,-5.65) {ACK $l+1$};
			\node at (5.25,-6) {închis};
		\end{tikzpicture}
		\caption{Strângere de mână în trei al TCP la conectare și deconectare}
		\label{tcp3way_ro}
	\end{figure}
	
	Figura \ref{tcp3way_ro} prezintă strângerea de mână în trei al protocolului TCP la procesul de conectare urmat de procesul de deconectare. După ce aplicația cheamă funcția \mintinline{cpp}{connect()} al sistemului, sistemul de operare va începe procesul cerut într-un mod asincron, și va returna cu o eroare sau un soclu deschis.
	
	După o conectare cu succes, aplicația va chema funcția \mintinline{cpp}{close()}, care va începe procesul de a încheia conexiunea, altfel ambele terți sunt expuse la problema de epuizarea resurselor, o problemă altfel categorizată în categoria de atacuri\cite{erickson08}.
	
	Sunt metode alternative pentru scanarea porturilor TCP, de exemplu cele bazate pe 'SYN', `ACK' și `FIN'\cite{kris07}, care folosesc socluri de tip ``raw'' pentru crearea și trimiterea pachetelor la destinatari, analizând răspunsul (sau lipsa răspunsului) în final.
	
	Aceste metode sunt folosite în general pentru a evada logurile și a evita paravanele de protecție. Aplicația realizată în cadrul tezei este orientată spre utilizatori care sunt autorizate. Din acest motiv, aplicația nu conține metodele alternative menționate în paragraful anterior.

\subsubsection*{Scanare UDP}

	Protocolul UDP, contrar protocolului TCP, oferă comunicație fără conexiune. Din acest motiv, nu există o metodă sigură pentru a determina dacă există o serviciu în spatele unui port, sau nu. Chiar dacă este o aplicație asociată la portul UDP scanat, e posibil ca aceasta să nu răspundă la mesajele care nu sunt formatate corect corespunzând specificațiilor al protocolului folosit de server. Cazul preferabil este ca serviciul să răspundă cu un mesaj de eroare, care poate fi analizat și folosit la identificarea protocolului și potențial aplicației.
	
	Anumite sisteme răspund cu un ICMP `PortUnreachable' pachet în cazul în care în spatele portul UDP scanat nu există nici o aplicație asociată, dar majoritatea firewall-urilor filtrează mesajele de acest tip în modul explicit pentru a combate posibilitatea de a scana porturile.
	
	Metoda cea mai bună pentru descoperea serviciilor pe porturile UDP e ca să trimitem mesaje la care este garantat că serverul (dacă rulează în spatele portului contactat) va răspunde. Componentul \mintinline{cpp}{UdpScanner} conține o bază de date alcătuit de numere de porturi asociate cu pachete care conțin un mesaj corect format al specificațiilor protocolului care e în general în spatele portului asociat. Ca un exemplu, fragmentul de cod \ref{dnsverreq_ro} prezintă reprezentația hexazecimală al unui pachet UDP generat de \mintinline{bash}{xxd}, care când e trimis pe portul 53 al unui gazdă care rulează un server DNS pe acest port, aceasta va răspunde cu versiunea aplicației.
	
	\begin{listing}[H]
		\begin{minted}{py}
			00000000: 34ef 0100 0001 0000 0000 0000 0756 4552  4............VER
			00000010: 5349 4f4e 0442 494e 4400 0010 0003       SION.BIND.....
		\end{minted}
		\caption{Pachet UDP pentru solicitarea versiunii unei server DNS}
		\label{dnsverreq_ro}
	\end{listing}
	
	Din păcate această metodă limitează lista de servicii detectabile la lista de servicii la care există un pachet de exemplu asociat în baza de date al componentei. Această metodă nu va funcționa nici în cazul în care serviciul a fost mutat de pe portul în care ar funcționa în mod implicit.
	
	În cazul în care utilizatorul solicită scanarea a unui port care nu are un pachet asociat în baza de date al componentului, aplicația generează un pachet alcătuit de 16 octeți de \mintinline{matlab}{NULL}, dar acest pachet nu este garantat ca o să genereze un răspuns de la serverul în spatele portului scanat.
	
	Dacă vrem să descoperim numai gazdele care sunt online într-o rețea, un răspuns de tip ICMP `PortUnreachable' la un mesaj UDP la un port la întâmplare înseamnă că gazda e online, numai portul nu e. În caz contrar, un răspuns de tip ICMP `HostUnreachable' înseamnă că gazda nu e online sau accesibilă. În general pachetul menționat anterior este generat de un router pe ruta între utilizator și rețeaua destinată.

\subsection*{Scanare Pasivă}

	Acest capitol prezintă componentul care realizează funcția de a iniția scanări pasive, prin care se înțelege o scanare care nu trimite pachete către gazdele scanate, deci scanarea astfel nu este detectabil de țintele.
	
	Sunt servicii care scanează în continuu toate adresele publice al protocolului IPv4, și publică rezultatele într-un format accesibil. Astfel, când se solicită o scanare pasivă, aplicația nu trimite pachete către serverele numite, ci descarcă datele disponibile pentru adresele specificate de la serviciile menționate anterior.
	
	Servicii există și pentru protocolul IPv6, dar din cauza numărului mărit de adrese de la $ 2^{32} $ până la $ 2^{128} $, aceste servicii nu pot să scaneze toate adresele, ci trebuie să folosească trucuri creative pentru a obține liste cu adrese active.
	
	Aplicația momentan suportă serviciile Shodan\cite{shodan16}, Censys\cite{censys15} și Mr Looquer\cite{looquer16}.
	
	Pentru a folosi aceste servicii, utilizatorul trebuie să se înregistreze la adresa fiecărui serviciu, și să obțină cheile de acces pentru API-urile fiecăruia, după ce îl poate furniza aceste chei prin argumentele \mintinline{bash}{--[shodan|censys|looquer]-key} în linia de comandă.
	
	Utilizatorul are posibilitatea de a folosi toate trei servicii concomitent, o funcție realizată în componentul \mintinline{cpp}{PassiveScanner}, care descarcă datele pentru adresele solicitate de la toate trei servicii.
	
	Amalgamarea datelor este o funcție utilă, fiindcă în majoritatea cazurilor nu toate serviciile au date disponibile pentru fiecare adresă, sau în câteva cazuri mai obscure, unele dintre servicii nu au date utile din cauza a unei restricții impuse de administratorul serverului.

\subsection*{Scanare Externală}

	Acest capitol prezintă componentul \mintinline{cpp}{NmapScanner}, care lansează o aplicație externală pentru a scana o gazdă și citește rezultatul scanării de la aplicația externală.
	
	Deși aplicația realizată în cadrul licenței are componente implementate pentru scanări active, ele fiind detaliate de la secția \ref{icmpping} până la secția \ref{udpscan}, acest component asigură o redundanță în cazul în care utilizatorul vrea o a doua opinie.
	
	Componentul suportă nativ rularea aplicației \textit{nmap}, dar se poate folosi oricare aplicație care generează un report de XML compatibil cu cel generat de nmap. Ca exemplu, o alternativă ar fi \textit{masscan}, pentru aplicațiile care generează reporturi compatibile și utilizabile.
	
	Pentru folosirea aplicației nmap, executabilul \mintinline{batch}{nmap} trebuie să fie accesibilă din variabila de mediu \mintinline{batch}{PATH}.

\subsection*{Identificare Software cu Expresie Regulată}

	Componentul \mintinline{cpp}{ServiceRegexMatcher} primește ca input un banner de serviciu, și îl testează împotriva unei baze de date conținând expresii regulate asociate cu nume CPE.
	
	\begin{listing}[H]
		\begin{minted}{matlab}
			^HTTP/1\.[01] \d{3}.*\r\nServer: nginx(?:/([\d.]+))?
		\end{minted}
		\caption{Expresie regulată pentru \mintinline{matlab}{cpe:/a:nginx:nginx}}
		\label{nginxregex_ro}
	\end{listing}
	
	Fragmentul de cod \ref{nginxregex_ro} conține o expresie regulată care se potrivește pentru răspunsul generat de serverul de HTTP numit ``nginx''. Expresia regulată mai conține și un grup opțional, care încearcă să extragă versiunea aplicației. În cazul în care acest grup a potrivit, componentul returnează, de exemplu, \mintinline{matlab}{cpe:/a:nginx:nginx:1.9.12}, sau dacă expresia s-a potrivit numai parțial, componentul returnează numai \mintinline{matlab}{cpe:/a:nginx:nginx}.
	
	Această metodă diferă de la cel discutat în secțiunea \ref{matchcpe}, fiindcă acest component, printre altele, poate să identifice și aplicații care nu anunță versiunea.
	
\subsection*{Identificare Software pe Bază de CPE}

	Institutul \textit{National Institute of Standards and Technology} menține o bază de date sub numele de \textit{National Vulnerability Database}, care conține vulnerabilitățile aplicaților mai populare. Componentul prezentat în acest capitol, și anume \mintinline{cpp}{CpeDictionaryMatcher}, folosește baza de date menționat anterior pentru identificarea aplicațiilor.
	
	\textit{CPE} este o schemă standardizată pentru a numi hardware, aplicații și sisteme de operare\cite{cpe22}.
	
	Versiunea al 2.2 este \mintinline{matlab}{cpe:/tip:proprietar:produs:versiune:actualizări:ediție:limbă}, unde componentul `tip' poate să fie \mintinline{matlab}{h} pentru `hardware', \mintinline{matlab}{o} pentru `sistem de operare' și \mintinline{matlab}{a} pentru `aplicație'.
	
	Ca un exemplu, numele pentru ``nginx 1.3.9'' este \mintinline{matlab}{cpe:/a:igor_sysoev:nginx:1.3.9}.
	
	Numele CPE nu sunt listate în bannerele de serviciu, și nici nu există o metodă sigură pentru a extrage o nume CPE dintr-un protocol. Publicația \cite{shovat15} abordă această problemă, iar soluția prezentată de autori este cea folosită și în aplicația realizată în cadrul licenței.
	
	Componentul folosește dicționarul CPE publicat de NIST, pre-procesat pentru a-l converti într-un format mai avantajos la utilizarea de acest tip, și pentru a înlătura datele neutilizate din motive de performanță și economie de memorie. Fragmentul de cod \ref{ciscotokens_ro} prezintă structura încărcată în memorie pentru o nume de CPE.
	
	\begin{listing}[H]
		\begin{minted}[style=perldoc]{js}
			CpeEntry {
				// cpe:/o:cisco:ios
				"ProductSpecificTokens": ["cisco", "ios"],
				"Versions": [
					CpeVersionEntry {
						// cpe:/o:cisco:ios:12.2sxi
						"VersionNumber": "12.2",
						"VersionSpecificTokens": ["sxi"]
					}
					// [celelalte versiuni omise]
				]
			}
		\end{minted}
		\caption{Structura aproximativă în memorie pentru \mintinline{matlab}{cpe:/o:cisco:ios:12.2sxi}}
		\label{ciscotokens_ro}
	\end{listing}
	
	Prin procesul de identificare pe baza de CPE, structura prezentată cu token-urile al numelor CPE este reiterată în primul rând până cuvintele specifice al aplicației se potrivesc în totalitate. Dacă se găsește o potrivire completă, versiunile asociate aplicației sunt reiterate, cautând o potrivire. Dacă s-a găsit o potrivire și printre numerele de versiuni, dar există o listă de cuvinte specifice ale versiunii, și aceste trebuie să se potrivesc în totalitate ca numele CPE să fie returnat ca o potrivire.
		
\subsection*{Validarea Vulnerabilităților}

	Aplicația realizată suportă identificarea și folosirea al sistemelor de operare \textbf{Debian}, \textbf{Ubuntu}, \textbf{Red Hat}, \textbf{CentOS} și \textbf{Fedora}, cu suport parțial pentru \textbf{Windows}.
	
	După scanare și analiza bannerelor de serviciu, aplicația încearcă să identifice sistemul de operare, și versiunea al sistemului de operare pentru fiecare gazdă scanată. Pentru sistemele menționate anterior cu suport complet, aplicația poate să valideze vulnerabilitățile descoperite prin rezolvarea numelor CPE identificate în nume actuale de pachete în distribuție, urmat de descărcarea al informațiilor disponibile pentru CVE-urile descoperite pentru pachetele rezolvate.
	
	Informațiile folosite pentru validarea vulnerabilităților astfel provin de la echipa de securitate al sistemului de operare sau de la baza de date de urmărirea erorilor al menținătorul pachetului din distribuție.

\section*{Rezultate}

	Pentru a testa aplicația implementată, am lansat o scanare în ambele moduri (activ și pasiv) contra trei universități. Am mai lansat încă o scanare în modul pasiv contra cinci instituții financiare, unde mă așteptam ca aceste instituții să reprezinte standardul ``gold'' în test.

	\begin{table}[H]
		\centering
		\begin{tabular}{r|ccc|ccc|ccccc|}
			\cline{2-12}
			\multicolumn{1}{l|}{}                         & \multicolumn{3}{c|}{\textbf{S. Activă}} & \multicolumn{8}{c|}{\textbf{Scanare Pasivă}}                                                             \\ \hline
			\multicolumn{1}{|r|}{\textbf{Instituția}}      & \textbf{$u_1$}    & \textbf{$u_2$}    & \textbf{$u_3$}   & \textbf{$u_1$} & \textbf{$u_2$} & \textbf{$u_3$} & \textbf{$b_1$} & \textbf{$b_2$} & \textbf{$b_3$} & \textbf{$b_4$} & \textbf{$b_5$} \\ \hline
			\multicolumn{1}{|r|}{\textbf{Servicii}} & 165            & 178            & 455           & 201         & 269         & 623         & 41          & 19          & 69          & 31          & 11          \\
			\multicolumn{1}{|r|}{\textbf{Identificabil}} & 143            & 145            & 402           & 112         & 148         & 352         & 24          & 8           & 58          & 9           & 9           \\
			\multicolumn{1}{|r|}{\textbf{Identificat}}   & 140            & 137            & 394           & 109         & 139         & 348         & 24          & 8           & 58          & 9           & 9           \\ \hline
		\end{tabular}
		\caption{Nume de CPE identificabile și identificate de aplicația realizată}
		\label{cpeids_ro}
	\end{table}
	
	Tabelul \ref{cpeids_ro} prezintă numărul serviciilor descoperite și identificate. Un ``serviciu'' reprezintă un port deschis pe gazdă. Rândul ``Identificabil'' reprezintă numărul serviciilor care au fost descoperite, și au trimis un service banner care să conțină informații utilizabile la identificare, de exemplu numele și versiunea softului de server. În ultimul rând, ``Identificat'' reprezintă numărul serviciilor care au fost identificate cu succes de aplicația realizată.
	
	Din cele \textbf{1,410} service banner-uri valide colectate, aplicația a identificat \textbf{1,374} servicii corect, care reprezintă o rată de succes de \textbf{97.45\%}.
	
	\begin{table}[H]
		\centering
		\begin{tabular}{r|ccc|ccc|ccccc|}
			\cline{2-12}
			\multicolumn{1}{l|}{}                         & \multicolumn{3}{c|}{\textbf{S. Activă}} & \multicolumn{8}{c|}{\textbf{Scanare Pasivă}}                                                             \\ \hline
			\multicolumn{1}{|r|}{\textbf{Instituția}}      & \textbf{$u_1$}    & \textbf{$u_2$}    & \textbf{$u_3$}   & \textbf{$u_1$} & \textbf{$u_2$} & \textbf{$u_3$} & \textbf{$b_1$} & \textbf{$b_2$} & \textbf{$b_3$} & \textbf{$b_4$} & \textbf{$b_5$} \\
			\multicolumn{1}{|r|}{\textbf{Servicii}} & 165            & 178            & 455           & 201         & 269         & 623         & 41          & 19          & 69          & 31          & 11          \\ \hline
			\multicolumn{1}{|r|}{\textbf{Critic}}       & 183            & 121            & 160           & 161         & 131         & 230         & 8           & 0           & 30          & 6           & 6           \\
			\multicolumn{1}{|r|}{\textbf{Înalt}}          & 675            & 414            & 645           & 583         & 446         & 826         & 7           & 0           & 67          & 21          & 5           \\
			\multicolumn{1}{|r|}{\textbf{Mediu}}        & 2545           & 1441           & 2775          & 2308        & 1589        & 3247        & 19          & 0           & 299         & 133         & 9           \\
			\multicolumn{1}{|r|}{\textbf{Scăzut}}       & 231            & 151            & 275           & 195         & 170         & 335         & 7           & 0           & 26          & 13          & 4           \\
			\multicolumn{1}{|r|}{\textbf{AV:N}}           & 3296           & 1970           & 3493          & 2961        & 2173        & 4248        & 40          & 0           & 393         & 153         & 22          \\ \hline
		\end{tabular}
		\caption{Numărul vulnerabilităților descoperite pentru serviciile identificate}
		\label{cpevulns_ro}
	\end{table}
	
	Tabelul \ref{cpevulns_ro} prezintă vulnerabilitățile descoperite pentru serviciile identificate cu succes de aplicația realizată. Rândurile ``Critic'', ``Înalt'', ``Mediu'' și ``Scăzut'' reprezintă severitatea a fiecărei vulnerabilități descoperite, iar rândul ``AV:N'' reprezintă numărul vulnerabilităților care pot fi abuzate prin rețea.
	
	\begin{table}[H]
		\centering
		\begin{tabular}{|r|l|}
			\hline
			\multicolumn{1}{|c|}{\textbf{Software}} & \multicolumn{1}{c|}{\textbf{CVE\#}} \\ \hline
			\textit{Host Scanner\footnotemark{}}                   & 166                                 \\
			OpenVAS                                 & 107                                 \\
			Nessus                                  & 68                                  \\
			Nexpose                                 & 311                                 \\ \hline
		\end{tabular}
		\caption{Comparație între vulnerabilitățile descoperite de diverse aplicații}
		\label{foundvulns_ro}
	\end{table}
	\footnotetext{Numele aplicației realizată în cadrul tezei.}
	
	Tabelul \ref{foundvulns_ro} prezintă o comparație între aplicația realizată în cadrul tezei și diversele aplicații comerciale prezentate în secțiunea \ref{comtools}. Aplicațiile au fost lansate contra o rețea virtuală alcătuită de diverse mașini virtuale folosite pentru pregătirile la concursuri de tip CTF (``Capture the Flag'').
	
	Aplicațiile ``OpenVAS'' și ``Nessus'' au identificat mai puține vulnerabilități, fiindcă nu au avut extensii pentru identificarea a serviciilor mai obscure pe gazdele din rețeaua scanată.
	
	În caz contrar, ``Nexpose'' a abuzat una dintre vulnerabilitățile folosite și a obținut acces shell. Având acces, aplicația a folosit managerul de pachete al sistemului de operare pentru a obține o listă cu numele și versiunea aplicațiilor instalate pe gazdă. Așa, a ajuns să descopere vulnerabilități care nu sunt accesibile/abuzabile de pe rețea, numai dacă utilizatorul are acces local.
	
	Aplicația realizată nu are ca scop abuzarea vulnerabilităților, fiindcă acest comportament nu este în general preferabil în cazul sistemelor în producție și e ilegal pentru cercetători în domeniul de securitate fără autorizație. Validarea vulnerabilităților se face prin metoda prezentată în secțiunea \ref{vulnvalid}, în scurt prin identificarea sistemului de operare urmat de folosirea managerului de pachete a distribuției identificată, fără penetrare.
