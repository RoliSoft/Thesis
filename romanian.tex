% !TeX spellcheck = ro_RO

\renewcommand{\listoflistingscaption}{Lista fragmente de coduri}
\renewcommand{\listingscaption}{Fragment de cod}

\newpage
\pagestyle{empty}
\selectlanguage{romanian}

	\begin{center}
		{\Large Universitatea Sapientia, Târgu-Mureș}\\\vspace{0.05in}
		{\Large Facultatea de Științe Tehnice și Umaniste}\\\vspace{0.07in}
		{\Large Calculatoare}\\
		
		\vspace{2.35in}
		
		{\huge Sistem pentru Testarea Penetrabilității și}\\\vspace{0.15in}
		{\huge Descoperirea Vulnerabilităților}
		
		\vspace{0.5in}
		
		{\LARGE Teză de Licență -- Extras}
		
	\end{center}
	
	\vspace{2.0in}
	
	\begin{multicols}{2}
		\begin{flushleft}
			{\Large Îndrumător Științific:}
		\end{flushleft}
		\columnbreak
		\begin{flushright}
			{\Large Absolvent:}
		\end{flushright}
	\end{multicols}
	\begin{multicols}{2}
		\begin{flushleft}
			{\LARGE Dr. Vajda Tamás}
		\end{flushleft}
		\columnbreak
		\begin{flushright}
			{\LARGE Bogosi Roland}
		\end{flushright}
	\end{multicols}
	
	\vspace{1.5in}
		
	\begin{center}
		{\LARGE 2016}
	\end{center}

\newpage

	\includepdf[pages=-]{declaration.pdf}

\newpage
\section*{Cuprins}

	\begingroup
	\renewcommand{\section}[2]{}
	\hypersetup{linkcolor=lightblue}
	\setlength{\parskip}{0em}
	\romaniantableofcontents
	\endgroup

\newpage
\section*{Scopul Proiectului}

	Scopul proiectului este de a dezvolta o aplicație, care poate să scaneze o rețea și să descoperă vulnerabilitățile rețelei scanate.

	Aplicația trebuie să fie complet autonomă în procesul de descoperirea vulnerabilităților, fără a depinde de actualizări de la dezvoltator pentru a identifica servicii și vulnerabilități noi.

	Publicul țintă al aplicației trebuie să fie cât mai largă. Aplicația trebuie să fie un instrument util pentru cercetători în domeniul de securitate, pentru consultanți, dar și pentru administratori, chiar în cazul în care administratorii rețelei nu au experiențe semnificative în domeniul de securitate.

\subsection*{Achiziție de Date}

	Pentru a fi folositor cercetătorilor independenți în domeniul de securitate, aplicația trebuie să conține metode de ``OSINT'' (``open-source intelligence''). Serviciile prezentate în secțiunea \ref{passive_ro}, anume Shodan\cite{shodan16}, Censys\cite{censys15} și Mr Looquer\cite{looquer16}, sunt candidate perfecți pentru componentul al aplicației care achiziționează date prin metode pasive.
	
	Prin suportarea metodelor pentru a achiziționa date fără o scanare activă, cercetătorii în domeniul de securitate pot să lanseze o scanare pentru care vor primi rezultate instante, fără necesitatea de a avea o infrastructură largă pentru scanare.
	
	În cazul contrar, consultanții în domeniul de securitate în general trebuie să lanseze o scanare împotriva unei infrastructuri private, pentru care date nu sunt disponibile pe serviciile menționate anterior. Din acest motiv, aplicația trebuie să conține și metode active pentru achiziționarea datelor.
	
	Indiferent de faptul că aplicația implementează mai multe moduri de a achiziționa date, ca un mijloc de redundanță, utilizatorii pot să lanseze scanarea cu o aplicație exterioară, importând rezultatele pentru procesare în final. În mod implicit, aplicația de scanare \textit{nmap} este suportat, dar orice aplicație se poate folosi care generează reporturi în format XML care sunt compatibile cu nmap; un exemplu pentru un astfel de aplicație ar fi \textit{masscan}.

\subsection*{Analizarea Datelor}

	Bannerele de serviciu (``service banner'') care au fost colectate trebuie să fie analizate într-un mod complet autonom. Contrar cu aplicațiile prezentate în secțiunea \ref{relwork_ro}, aplicația dezvoltată nu trebuie să depinde pe extensii pentru a identifica servicii și descoperi vulnerabilități.

	Aplicația trebuie să folosească bazele de date publicate de instituția \textit{NIST}, discutat în secțiunea \ref{vulndbs_ro}. Secțiunea \ref{matchcpe_ro} prezintă componentele care procesează bannerele de serviciu folosind bazele de date menționate anterior, și asociază o nume \textit{CPE} într-un mod complet autonom.

	Ca un mijloc de redundanță, secțiunea \ref{patternmatch_ro} prezintă o metodă alternativă pentru procesarea bannerelor de serviciu. Această metodă nu folosește bazele de date menționate anterior, ci o bază de date care conține nume CPE asociate cu expresii regulate, creat și menținut de dezvoltatorul aplicației sau comunitatea.

\subsection*{Sugestii de Soluționare}

	După ce serviciile scanate au fost identificate, descoperirea vulnerabilităților este o simplă operațiune de căutarea în baza de date cu vulnerabilități \textit{CVE}.
	
	Din punctul acesta, aplicația trebuie să prezintă cât mai multe informații utile utilizatorului despre serviciile și vulnerabilitățile descoperite, ca și cum ar fi punctajul de CVSS al vulnerabilităților în descoperite în infrastructură pentru analiza de risc.
	
	Ca să fie folositoare și pentru administratorii de rețea care nu au experiențe semnificative în domeniul de securitate, aplicația trebuie să încearcă să generează o listă simplă cu pașii pentru a soluționa vulnerabilitățile descoperite pe fiecare server.
	
	În cazul serverurilor unde sistemul de operare a fost identificat cu succes, aplicația poate să generează o linie de comandă care când este rulat pe serverul cu vulnerabilități va actualiza programele vulnerabile.

\section*{Soluții Aferente} \label{relwork_ro}
	
\subsection*{Soluții Comerciale} \label{comtools_ro}
	
	Proiectul \textit{nmap} menține o listă cu instrumentele de securitate, ordonat pe baza popularității acestora\cite{sectools}. În această secțiune vor fi prezentate trei instrumente de penetrare dintre cele mai populare din lista menționată ulterior.
	
	\paragraph*{Nessus} \textit{Nessus Vulnerability Scanner}\cite{nessus} este dezvoltat și menținut de Tenable Network Security. Inițial a fost gratuit și open-source, cu funcții opționale contra cost, dar în 2005 sursa a fost închisă. SecTools.org pune Nessus pe poziția numărul unu pe popularitatea instrumentelor din această categorie. Momentan există o ediție ``Community'' care e gratuit, dar are câteva funcții limitate și se poate folosi numai pentru uz personal. Abonamentul anual pentru versiunea plătită începe de la \$2,190 per utilizator.
	
	\paragraph*{OpenVAS} \textit{Open Vulnerability Assessment System}\cite{openvas} este un fork al proiectului Nessus, din perioada în care era open-source. Momentan este dezvoltat și menținut de Greenbone Networks. SecTools.org pune OpenVAS pe poziția numărul doi pe popularitate instrumentelor din această categorie, iar popularitatea acestuia este într-o creștere continuară din cauza sursei deschise.
	
	\paragraph*{Nexpose} \textit{Nexpose Vulnerability Scanner}\cite{nexpose} este dezvoltat și menținut de Rapid7, o companie care este cunoscut în domeniul de securitate din cauza unui alt proiectul dezvoltate de ei, anume \textit{Metasploit}. SecTools.org pune Nexpose pe poziția numărul trei pe popularitate instrumentelor din această categorie. Abonamentele anuale încep de la \$2,000, dar există și o variantă ``Community'' care se poate folosii gratis pentru uz personal.
	
\subsection*{Lucrări Științifice}
	
	Anumite articole studiază eficiența al soluțiilor comerciale existente. Cele mai citate dintre aceste articole sunt \cite{holm11,bau10,doupe10}.
	
	Publicația \cite{kals06} detaliază planificarea și realizarea unei sistem de teste de penetrare pentru web, iar dintr-o altă perspectivă, articolul \cite{guo05} studiază securitatea dispozitivelor conectate la o rețea publică.
	
	\textit{ShoVAT}\cite{shovat15} prezintă o metodologie care este folosită și în aplicația dezvoltată în scopul licenței cu mici modificări, componentul fiind detaliat în secțiunea \ref{matchcpe_ro}, vorbind despre identificarea serviciilor într-un mod autonom, folosind baza de date publicat de instituția NIST.

\section*{Rezultate}

	Pentru a testa aplicația implementată, am lansat o scanare în ambele moduri (activ și pasiv) contra trei universități. Am mai lansat încă o scanare în modul pasiv contra cinci instituții financiare, unde mă așteptam ca aceste instituții să reprezinte standardul ``gold'' în test.

	\begin{table}[H]
		\centering
		\begin{tabular}{r|ccc|ccc|ccccc|}
			\cline{2-12}
			\multicolumn{1}{l|}{}                         & \multicolumn{3}{c|}{\textbf{S. Activă}} & \multicolumn{8}{c|}{\textbf{Scanare Pasivă}}                                                             \\ \hline
			\multicolumn{1}{|r|}{\textbf{Instituția}}      & \textbf{$u_1$}    & \textbf{$u_2$}    & \textbf{$u_3$}   & \textbf{$u_1$} & \textbf{$u_2$} & \textbf{$u_3$} & \textbf{$b_1$} & \textbf{$b_2$} & \textbf{$b_3$} & \textbf{$b_4$} & \textbf{$b_5$} \\ \hline
			\multicolumn{1}{|r|}{\textbf{Servicii}} & 165            & 178            & 455           & 201         & 269         & 623         & 41          & 19          & 69          & 31          & 11          \\
			\multicolumn{1}{|r|}{\textbf{Identificabil}} & 143            & 145            & 402           & 112         & 148         & 352         & 24          & 8           & 58          & 9           & 9           \\
			\multicolumn{1}{|r|}{\textbf{Identificat}}   & 140            & 137            & 394           & 109         & 139         & 348         & 24          & 8           & 58          & 9           & 9           \\ \hline
		\end{tabular}
		\caption{Nume de CPE identificabile și identificate de aplicația realizată}
		\label{cpeids_ro}
	\end{table}
	
	Tabelul \ref{cpeids_ro} prezintă numărul serviciilor descoperite și identificate. Un ``serviciu'' reprezintă un port deschis pe gazdă. Rândul ``Identificabil'' reprezintă numărul serviciilor care au fost descoperite, și au trimis un service banner care să conține informații utilizabile la identificare, de exemplu numele și versiunea softului de server. În ultimul rând, ``Identificat'' reprezintă numărul serviciilor care au fost identificate cu succes de aplicația realizată.
	
	Din cele \textbf{1,410} service banner-uri valide colectate, aplicația a identificat \textbf{1,374} servicii corect, care reprezintă o rată de succes de \textbf{97.45\%}.
	
	\begin{table}[H]
		\centering
		\begin{tabular}{r|ccc|ccc|ccccc|}
			\cline{2-12}
			\multicolumn{1}{l|}{}                         & \multicolumn{3}{c|}{\textbf{S. Activă}} & \multicolumn{8}{c|}{\textbf{Scanare Pasivă}}                                                             \\ \hline
			\multicolumn{1}{|r|}{\textbf{Instituția}}      & \textbf{$u_1$}    & \textbf{$u_2$}    & \textbf{$u_3$}   & \textbf{$u_1$} & \textbf{$u_2$} & \textbf{$u_3$} & \textbf{$b_1$} & \textbf{$b_2$} & \textbf{$b_3$} & \textbf{$b_4$} & \textbf{$b_5$} \\
			\multicolumn{1}{|r|}{\textbf{Servicii}} & 165            & 178            & 455           & 201         & 269         & 623         & 41          & 19          & 69          & 31          & 11          \\ \hline
			\multicolumn{1}{|r|}{\textbf{Critic}}       & 183            & 121            & 160           & 161         & 131         & 230         & 8           & 0           & 30          & 6           & 6           \\
			\multicolumn{1}{|r|}{\textbf{Înalt}}          & 675            & 414            & 645           & 583         & 446         & 826         & 7           & 0           & 67          & 21          & 5           \\
			\multicolumn{1}{|r|}{\textbf{Mediu}}        & 2545           & 1441           & 2775          & 2308        & 1589        & 3247        & 19          & 0           & 299         & 133         & 9           \\
			\multicolumn{1}{|r|}{\textbf{Scăzut}}       & 231            & 151            & 275           & 195         & 170         & 335         & 7           & 0           & 26          & 13          & 4           \\
			\multicolumn{1}{|r|}{\textbf{AV:N}}           & 3296           & 1970           & 3493          & 2961        & 2173        & 4248        & 40          & 0           & 393         & 153         & 22          \\ \hline
		\end{tabular}
		\caption{Numărul vulnerabilităților descoperite pentru serviciile identificate}
		\label{cpevulns_ro}
	\end{table}
	
	Tabelul \ref{cpevulns_ro} prezintă vulnerabilitățile descoperite pentru serviciile identificate cu succes de aplicația realizată. Rândurile ``Critic'', ``Înalt'', ``Mediu'' și ``Scăzut'' reprezintă severitatea a fiecărei vulnerabilități descoperite, iar rândul ``AV:N'' reprezintă numărul vulnerabilităților care pot fi abuzate prin rețea.
	
	\begin{table}[H]
		\centering
		\begin{tabular}{|r|l|}
			\hline
			\multicolumn{1}{|c|}{\textbf{Software}} & \multicolumn{1}{c|}{\textbf{CVE\#}} \\ \hline
			\textit{Host Scanner\footnotemark{}}                   & 166                                 \\
			OpenVAS                                 & 107                                 \\
			Nessus                                  & 68                                  \\
			Nexpose                                 & 311                                 \\ \hline
		\end{tabular}
		\caption{Comparație între vulnerabilitățile descoperite de diverse aplicații}
		\label{foundvulns_ro}
	\end{table}
	\footnotetext{Numele aplicației realizată în cadrul tezei.}
	
	Tabelul \ref{foundvulns_ro} prezintă o comparație între diverse aplicații comerciale și aplicația realizată în cadrul tezei. Aplicațiile au fost lansate contra o rețea virtuală alcătuită de diverse mașini virtuale folosite pentru pregătirile la concursuri de tip CTF (``Capture the Flag'').
	
	Aplicațiile ``OpenVAS'' și ``Nessus'' au identificat mai puține vulnerabilități, fiindcă nu au avut extensii pentru a identifica câteva servicii mai obscure pe gazdele în rețeaua scanată.
	
	În caz contrar, ``Nexpose'' a abuzat una dintre vulnerabilitățile folosite și a obținut acces shell. Având acces, aplicația a folosit managerul de pachete al sistemului de operare pentru a obține o listă cu numele și versiunea aplicațiilor instalate pe gazdă. Așa, a ajuns să descopere vulnerabilități care nu sunt accesibile/abuzabile de pe rețea, numai dacă utilizatorul are acces local.
	
	Aplicația realizată nu are ca scop abuzarea vulnerabilităților, fiindcă acest comportament nu este în general preferabil în cazul sistemelor în producție și e ilegal pentru cercetători în domeniul de securitate fără autorizație. Validarea vulnerabilităților se face prin identificarea sistemului de operare urmat de folosirea managerului de pachete a distribuției identificată, fără penetrare.
