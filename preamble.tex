% !TeX spellcheck = en_US

\usepackage[utf8x]{inputenc}
\usepackage[margin=1in,headheight=14.5pt]{geometry}
%\usepackage[top=1in,bottom=1in,right=0.7in,left=1.3in,headheight=14.5pt]{geometry}

\usepackage{fancyhdr}
\pagestyle{fancy}

\usepackage[english,romanian,hungarian]{babel}
\usepackage{indentfirst}

\usepackage[table,xcdraw,dvipsnames]{xcolor}
\definecolor{whitesmoke}{rgb}{0.96,0.96,0.96}
\definecolor{lightblue}{rgb}{0.22,0.45,0.70}
\definecolor{darkred}{rgb}{0.9,0.0,0.0}
\definecolor{darkpastelgreen}{rgb}{0.01,0.75,0.24}
\definecolor{flame}{rgb}{0.89,0.35,0.13}

\usepackage[unicode,pdfusetitle,backref=false,pagebackref=true]{hyperref}
%\hypersetup{colorlinks=true,urlcolor=lightblue,citecolor=darkpastelgreen,linkcolor=flame,pdfborder={0 0 0}}
\hypersetup{colorlinks=true,urlcolor=black,citecolor=black,linkcolor=black,pdfborder={0 0 0}}
\usepackage[all]{hypcap}
\usepackage{multirow}
\usepackage{letltxmacro}

\usepackage{graphicx}
\usepackage{wrapfig}
\usepackage{pdfpages}

\usepackage{booktabs}
\usepackage{multicol}
\usepackage{amsmath}

\usepackage[scale=1.25]{ccicons}

\usepackage{tikz}
\usetikzlibrary{arrows}
\tikzstyle{es} = [-triangle 60]

\usepackage{relsize}
\usepackage{enumitem}
\setlist[itemize]{itemsep=0pt}
\setlist[enumerate]{itemsep=0pt}

\setlength{\parskip}{0.25em}

\renewcommand{\arraystretch}{1.25}

\linespread{1.15}

\renewcommand{\UrlFont}{\ttfamily\relsize{-0.5}}

\usepackage[T1]{fontenc}
\usepackage{lmodern}
\usepackage{fontawesome}
\usepackage{inconsolata}

\usepackage{minted}
\renewcommand{\theFancyVerbLine}{\rmfamily\scriptsize\arabic{FancyVerbLine}}
\setminted{linenos,autogobble,breaklines,fontsize=\fontsize{10.5}{12.6},tabsize=4,numbersep=7pt,bgcolor=whitesmoke}
\AtBeginEnvironment{minted}{%
	\renewcommand{\fcolorbox}[4][]{#4}%
	\linespread{1.1}%
}

%
%	Override \cite and \ref to use bold face instead of a color.
%

\makeatletter
\def\@cite#1#2{[\textbf{#1\if@tempswa , #2\fi}]}
\makeatother

\AtBeginDocument{%
	\LetLtxMacro{\latexref}{\ref}%
	\def\ref#1{\textbf{\latexref{#1}}}%
}

%
%	Add "Retrieved from" and backlinks stylized by red arrows to the bibliography.
%

\newcommand{\urlprefix}{Retrieved from \urlstyle{rm}}

\newcommand{\refspace}{\vspace{-2mm}}
\newcommand{\redarrow}{\textbf{\textcolor{red}{$\mathbf{\to}$}}}

\makeatletter
\def\BR@@bibitem#1#2\par{
	\let\backrefprint\BR@backrefprint
	\def\@linkcolor{black}
	\BRorg@bibitem{#1}#2\redarrow \thinspace \BR@backref{#1}
}
\makeatother

%
%	Implement \subsubsubsection.
%

\usepackage{titlesec}

\titleclass{\subsubsubsection}{straight}[\subsection]

\newcounter{subsubsubsection}[subsubsection]
\renewcommand\thesubsubsubsection{\thesubsubsection.\arabic{subsubsubsection}}
\titleformat{\subsubsubsection}{\normalfont\normalsize\bfseries}{\thesubsubsubsection}{1em}{}
\titlespacing*{\subsubsubsection}{0pt}{3.25ex plus 1ex minus .2ex}{1.5ex plus .2ex}

\makeatletter
\def\toclevel@subsubsubsection{4}
\def\l@subsubsubsection{\@dottedtocline{4}{7em}{4em}}

\renewcommand\paragraph{\@startsection{paragraph}{6}{\parindent}{3.25ex \@plus1ex \@minus .2ex}{-1em}{\normalfont\normalsize\bfseries}}
\makeatother

\setcounter{secnumdepth}{4}
\setcounter{tocdepth}{4}

%
%	Implement Hungarian and Romanian section tracking for the English thesis.
%	\<lang>tableofcontents will print the corresponding one.
%

\makeatletter
\newcommand\hungariantableofcontents{
	\selectlanguage{hungarian}
	\section*{\contentsname
		\@mkboth{\MakeUppercase\contentsname}{\MakeUppercase\contentsname}}
	\@starttoc{2.toc}
	\selectlanguage{english}
}
\makeatother

\newcommand\sectionhu[1]{\addcontentsline{2.toc}{section} {\protect\numberline{\thesection} #1}}
\newcommand\subsectionhu[1]{\addcontentsline{2.toc}{subsection} {\protect\numberline{\thesubsection} #1}}
\newcommand\subsubsectionhu[1]{\addcontentsline{2.toc}{subsubsection} {\protect\numberline{\thesubsubsection} #1}}
\newcommand\subsubsubsectionhu[1]{\addcontentsline{2.toc}{subsubsubsection} {\protect\numberline{\thesubsubsubsection} #1}}

\makeatletter
\newcommand\romaniantableofcontents{
	\selectlanguage{romanian}
	\section*{\contentsname
		\@mkboth{\MakeUppercase\contentsname}{\MakeUppercase\contentsname}}
	\@starttoc{1.toc}
	\selectlanguage{english}
}
\makeatother

\newcommand\sectionro[1]{\addcontentsline{1.toc}{section} {\protect\numberline{\thesection} #1}}
\newcommand\subsectionro[1]{\addcontentsline{1.toc}{subsection} {\protect\numberline{\thesubsection} #1}}
\newcommand\subsubsectionro[1]{\addcontentsline{1.toc}{subsubsection} {\protect\numberline{\thesubsubsection} #1}}
\newcommand\subsubsubsectionro[1]{\addcontentsline{1.toc}{subsubsubsection} {\protect\numberline{\thesubsubsubsection} #1}}

%
%	Support for multiple \listof[figures|tables|listings] between parts of the document.
%	\clearcontents begins a new list.
%

\usepackage{xpatch}

\newcounter{lofcntr}
\newcounter{lotcntr}
\newcounter{lolcntr}

\NewDocumentCommand{\clearcontents}{}{
	\stepcounter{lofcntr}
	\stepcounter{lotcntr}
	\stepcounter{lolcntr}
	\setcounter{figure}{0}
	\setcounter{table}{0}
	\setcounter{listing}{0}
}

\makeatletter
\NewDocumentCommand{\advancecontents}{}{
	\def\ext@figure{\number\value{lofcntr}.\latex@ext@figure}
	\def\ext@table{\number\value{lotcntr}.\latex@ext@table}
	\@namedef{ext@listing}{\number\value{lolcntr}.\latex@ext@listing}
}

\AtBeginDocument{
	\let\latex@ext@figure\ext@figure
	\let\latex@ext@table\ext@table
	\let\latex@ext@listing\ext@listing
	\stepcounter{lofcntr}
	\stepcounter{lotcntr}
	\stepcounter{lolcntr}
	\advancecontents
}

\AtBeginDocument{
\xpretocmd{\caption}{
	\advancecontents
}{}{}
}

\xpatchcmd{\listoffigures}{%
	\@starttoc{lof}%
}{%
	\@starttoc{\number\value{lofcntr}.lof}%
}{}{}

\xpatchcmd{\listoftables}{%
	\@starttoc{lot}%
}{%
	\@starttoc{\number\value{lotcntr}.lot}%
}{}{}
\makeatother
