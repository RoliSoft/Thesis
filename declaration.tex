% !TeX spellcheck = ro_RO

\newpage
\pagestyle{plain}
\selectlanguage{romanian}

	\begin{center}
		{\Large Universitatea Sapientia din Cluj-Napoca}\\\vspace{0.07in}
		{\Large Facultatea de Științe Tehnice și Umaniste, Târgu-Mureș}\\\vspace{0.07in}
		{\Large Specializarea Calculatoare}\\
		
		\vspace{1.5cm}
		
		{\Large Declarație}
	\end{center}
	
	Subsemnatul, \emph{Bogosi Roland}, student la specializarea Calculatoare, Facultatea de Științe Tehnice și Umaniste din Târgu-Mureș, de la Universitatea Sapientia din Cluj-Napoca, certific că am luat la cunoștință de cele prezentate mai jos și îmi asum, în acest context, originalitatea lucrării mele de diplomă cu titlul \emph{Sistem pentru Testarea Penetrabilității și Descoperirea Vulnerabilităților}, coordonator \emph{dr. Vajda Tamás}, prezentată în sesiunea iunie 2016.
	
	\vspace{\fill}
	
	\noindent La elaborarea lucrării de diplomă, se consideră plagiat una dintre următoarele acțiuni:
	
	\begin{itemize}
		\item[--] reproducerea exactă a cuvintelor unui alt autor, dintr-o altă lucrare, în limba română sau prin traducere dintr-o altă limbă, dacă se omit ghilimelele și referința precisă;
		\item[--] redarea cu alte cuvinte, reformularea prin cuvinte proprii sau rezumarea ideilor din alte lucrări dacă nu se indică sursa bibliografică;
		\item[--] prezentarea unor date experimentale obținute sau a unor aplicații realizate de alți autori fără menționarea corectă a acestor surse;
		\item[--] însușirea totală sau parțială a unei lucrări în care regulile de mai sus sunt respectate, dar care are alt autor.
	\end{itemize}
	
	\begin{multicols}{2}
		\noindent Data: 27.06.2016
		\columnbreak
		\begin{center}
			Semnătura:
		\end{center}
	\end{multicols}
	
	\vspace{\fill}
	
	\noindent\emph{Notă:} Se recomandă:
	
	\begin{itemize}
		\item[--] plasarea între ghilimele a citatelor directe și indicarea referinței într-o listă corespunzătoare la sfârșitul lucrării;
		\item[--] indicarea în text a reformulării unei idei, opinii sau teorii și corespunzător în lista de referințe a sursei originale de la care s-a făcut preluarea;
		\item[--] precizarea sursei de la care s-au preluat date experimentale, descrieri tehnice, figuri, imagini, statistici, tabele etc.;
		\item[--] precizarea referințelor poate fi omisă dacă se folosesc informații sau teorii arhicunoscute, a căror paternitate este unanim acceptată.
	\end{itemize}
	
\newpage

	\begin{center}
		{\Large Universitatea Sapientia din Cluj-Napoca}\\\vspace{0.07in}
		{\Large Facultatea de Științe Tehnice și Umaniste, Târgu-Mureș}\\\vspace{0.07in}
		{\Large Specializarea Calculatoare}\\
	\end{center}
	
	\vspace{3cm}
	
	\begin{multicols}{2}
		\begin{center}
			Vizat decan\\
			Ș.l. Dr. ing. Kelemen András
		\end{center}
		\columnbreak
		\begin{center}
			Vizat șef catedră\\
			Dr. ing. Domokos József
		\end{center}
	\end{multicols}
