\documentclass[a4paper,12pt]{article}

\usepackage[utf8x]{inputenc}
\usepackage[english,hungarian]{babel}
\usepackage[margin=1in]{geometry}

\usepackage{indentfirst}
\usepackage{fancyhdr}
\pagestyle{fancy}

\usepackage[table,xcdraw,dvipsnames]{xcolor}
\definecolor{whitesmoke}{rgb}{0.96,0.96,0.96}
\definecolor{lightblue}{rgb}{0.22,0.45,0.70}
\definecolor{darkred}{rgb}{0.9,0.0,0.0}

\usepackage[backref=false,pagebackref=true]{hyperref}
\hypersetup{colorlinks=true,urlcolor=lightblue,citecolor=green!75!black,linkcolor=orange!98!black,pdfborder={0 0 0}}
\usepackage{multirow}

\usepackage{graphicx}
\usepackage{wrapfig}

\usepackage{booktabs}
\usepackage{multicol}
\usepackage{amsmath}

\usepackage{tikz}
\usetikzlibrary{arrows}
\tikzstyle{es} = [-triangle 60]

\usepackage{enumitem}
\setlist[itemize]{itemsep=0pt}
\setlist[enumerate]{itemsep=0pt}

\setlength{\parindent}{2em}
\setlength{\parskip}{0.25em}

\renewcommand{\arraystretch}{2}

\usepackage[T1]{fontenc}
\usepackage{lmodern}
\usepackage{fontawesome}

\newcommand{\urlprefix}{Retrieved from \urlstyle{rm}}

\newcommand{\refspace}{\vspace{-2mm}}
\newcommand{\redarrow}{\textcolor{darkred}{$\mathbf{\to}$}}
\makeatletter
\def\BR@@bibitem#1#2\par{
	\let\backrefprint\BR@backrefprint
	\def\@linkcolor{black}
	\BRorg@bibitem{#1}#2\redarrow \thinspace \BR@backref{#1}
}
\makeatother

\usepackage{titlesec}
\titleclass{\subsubsubsection}{straight}[\subsection]

\newcounter{subsubsubsection}[subsubsection]
\renewcommand\thesubsubsubsection{\thesubsubsection.\arabic{subsubsubsection}}
\titleformat{\subsubsubsection}
  {\normalfont\normalsize\bfseries}{\thesubsubsubsection}{1em}{}
\titlespacing*{\subsubsubsection}
{0pt}{3.25ex plus 1ex minus .2ex}{1.5ex plus .2ex}

\makeatletter
\def\toclevel@subsubsubsection{4}
\def\l@subsubsubsection{\@dottedtocline{4}{7em}{4em}}
\makeatother

\setcounter{secnumdepth}{4}
\setcounter{tocdepth}{4}

\usepackage{minted}
\renewcommand{\theFancyVerbLine}{\rmfamily\scriptsize\arabic{FancyVerbLine}}
\setminted{linenos,autogobble,breaklines,fontsize=\footnotesize,tabsize=4,numbersep=7pt,bgcolor=whitesmoke}
\AtBeginEnvironment{minted}{\renewcommand{\fcolorbox}[4][]{#4}}

\author{Bogosi Roland}
\title{Behatolástesztelő és sebezhetőségfelderítő rendszer}

\begin{document}
\pagestyle{empty}
\selectlanguage{hungarian}

	\begin{center}
		{\Large Sapientia Erdélyi Magyar Tudományegyetem}\\\vspace{0.05in}
		{\Large Műszaki és Humántudományok Kar, Marosvásárhely}\\\vspace{0.07in}
		{\Large Számítástechnika}\\
		
		\vspace{2.5in}
		
		{\huge Behatolástesztelő és Sebezhetőségfelderítő}\\\vspace{0.1in}
		{\huge Rendszer}
		
		\vspace{0.5in}
		
		{\LARGE TDK Dolgozat}
		
	\end{center}
	
	\vspace{2.0in}
	
	\begin{multicols}{2}
		\begin{flushleft}
			{\Large Vezető tanár:}\\\vspace{0.1in}
			{\LARGE {Dr. Vajda Tamás}}
		\end{flushleft}
		\columnbreak
		\begin{flushright}
			{\Large Diák:}\\\vspace{0.1in}
			{\LARGE {Bogosi Roland}}
		\end{flushright}
	\end{multicols}
	
	\vspace{1.5in}
		
	\begin{center}
		{\LARGE 2016}
	\end{center}

\newpage
\section*{Tartalomjegyzék}

	\begingroup
	\renewcommand{\section}[2]{}
	\hypersetup{linkcolor=lightblue}
	\setlength{\parskip}{0em}
	\tableofcontents
	\endgroup

	\begingroup
	\hypersetup{linkcolor=lightblue}
	\listoffigures
	%\listoftables
	\renewcommand*{\listoflistingscaption}{Kódrészletek jegyzéke}
	\listoflistings
	\endgroup

\newpage
\pagestyle{fancy}
\section{Bevezető}
	
	
	
\section{Szoftver Sebezhetőségek}
	
	Az \textit{RFC 2828} és számos \textit{NIST} publikáció úgy definiálja a ``sebezhetőséget,'' mint ``egy hiba vagy gyengeség a rendszer biztonsági procedúráiban, tervezésében, implementálásában, vagy belső ellenőrzései között amelyet ki lehet játszani (véletlenül kiváltva vagy szándékosan kihasználva) és így az eredmény egy biztonsági rés vagy a rendszer biztonságpolitikájának megsértése.''\cite{rfc2828,nist80030} Egyszerűen átfogalmazva ez annyit jelent, hogy a sebezhetőség egy hiba a szoftver vagy webszolgáltatás kódjában, amely kihasználáskor (például a felhasználó egy olyan bemenentet ad meg amely szándékosan úgy lett formázva, hogy kiváltsa az ismert hibát) megengedi a felhasználó számára, hogy olyan tevékenységet hajtson végre amelyt különben nem szabadna, vagy olyan információhoz férjen hozzá, amelyhez nem kéne.
	
\subsection{Közzétételi Eljárások és Politikák}
	
	Harmadik féltől származó szoftverekben talált hibák helyes jelentése és korrigálása egy komplikált téma, ugyanis a legtöbb szoftverfejlesztőcégnek nincs jól előre meghatározott és publikált eljárása biztonsági hibajelentés célokra. Egy standard, felelősségteljes közzétételi eljárás, hogy a hiba megtalálója kapcsolatba lép a szoftverfejlesztő céggel, és ezt követően egy bizonyos számú napot vár mielőtt a hibát publikusan is közzéteszi, a cég reakciójától függően.
	
	Egy másik probléma is felléphet abban az esetben ha a hiba több cég termékét is érinti, amely esetében előnyös egy koordonált közzétételt szervezni, ugyanis ellentkezőképpen az a cég amely a leghamarabb publikálja a biztonsági frissítést beleértve a rés információját, az veszélybe helyezi a többi cég klienseit amelyek számára még nem készült el a biztonsági frissítés.
	
	A biztonsági rést felfedező személy etikáitól és a rés veszélyességétől függően választani lehet az illegális utat is, amely során a biztonsági rést egy illegális marketen el lehet adni. A ``deep web''-en számos ilyen piactér van ahol úgynevezett ``nulladik napi'' sebezhetőségekkel kereskednek. Ezeknek az árai párszáz dollártól akár félmillió dollárig is filmehetnek egy New York Times-ban megjelent cikk szerint\cite{nperlroth13}.
	
	Ahhoz, hogy a magas profilú cégek ösztönözzék a biztonsági szakértőket, hogy a felelősségteljes közzétételi utat válasszák, sok esetben ``bug bounty'' programokat indítanak. Ezen programok célja, hogy publikáljanak egy közzétételi procedúrát a szakértők számára, illetve hogy jutalmazzák a bejelentéseket.
	
	Egy híres példa a Google \textit{Vulnerability Reward Program}\cite{googlevrp15} programja, amely akár többezer dollárt is fizet sebezhetőségekért a saját szolgáltatásukban, de később kibővítették a program által lefedett alkalmazások listáját pár nagy-kockázatú népszerű nyílt-forrású szerverre is, amelyek sebezhetőségeikért \$500-tól \$3,133.7-ig terjedő pénzjutalmat is felajánlanak. Ezenkívül a Google futtat egy \textit{Patch Reward Program} programot is, amelyen keresztül olyan önkénteseket jutalmaznak akik biztonsági réseket tömnek be nyílt-forrású alkalmazásokban. A jutalmak \$500-tól kezdődnek ``egy-soros javításokért'' és elérnek \$10,000-ig ``komplikált, nagy-hatású javítások'' esetén.
	
\subsection{Szabályozó Szervek}
	
	\begin{wrapfigure}{r}{0.35\textwidth}
		\vspace{-10pt}
		\centering
		\includegraphics[scale=0.75]{cert.png}
		\caption{CERT/CC Logó}
	\end{wrapfigure}
	
	Az Egyesült Államokban az egyik szabályozó szerv a \textit{CERT/CC} (\textit{Computer Emergency Response Team Coordination Center}) amely a \textit{DARPA} (\textit{Defense Advanced Research Projects Agency}) által volt alapítva 1988 novemberében. Az alapítás szükségességét az adta, hogy akkor készült el az első számítógépes vírus amely az Internet segítségével tovább tudott terjeszkedni, így megfertőzve egyéb számítógépeket. A vírus a \textit{Morris worm}\cite{cert15} nevet kapta, és nagy figyelemnek örvendett a média által.
	
	A CERT/CC-nek saját közzétételi politikája van: a szoftverfejlesztő céggel felveszik a kapcsolatot amint lehetséges, és ettől függetlenül a biztonsági rés leírása publikálva lesz 45 napon belül, akár a cég készített-e biztonsági frissítést, vagy sem.
	
	A szabályozó szerv Romániában a \textit{CERT-RO}, (\textit{Centrul Național de Răspuns la Incidente de Securitate Cibernetică}) amely 2011 májusában lett alapítva\cite{certro12}. Magyarországon ennek megfelelője a \textit{GovCERT-Hungary}, (\textit{Kormányzati Eseménykezelő Központ}) amely 2013 áprilisában lett alapítva\cite{certhu13}. Létezik egy szabályozó szerv az Európai Úniós szinten is: a \textit{CERT-EU} (\textit{Computer Emergency Response Team European Union Task Force}) 2012 novemberében volt alapítva\cite{certeu13}, és saját szabályozó szerepet tölt be, nem csak a tagországok szervei által küldött jelentéseket összesítik.
	
\subsection{Sebezhetőség Adatbázisok}
	
	A CERT/CC által egyik szponzorált projekt a \textit{CVE} (\textit{Common Vulnerabilities and Exposures}), amely egy módszert biztosít a publikusan ismert sebezhetőségek címkézésére és követésére. A \textit{NIST} (\textit{National Institute of Standards and Technology}) alapítvány futtat egy weboldalt \textit{NVD} (\textit{National Vulnerability Database}) néven, amelyen keresztül karbantartanak egy adatbázist amely a publikusan ismert sebezhetőségeket tartalmazza, illetve információkat ezekről olyan struktúrált formátumban, amelyet számítógépes alkalmazások is önműködően olvashatnak\cite{nvd15}.
	
	Az adatbázis, illetve egyéb komponenseinek felhasználása bővebben a sebezhetőségkereső komponens fejezetben van tárgyalva.
	
\section{Sebezhetőségek Felderítése}
	
	A \textit{sebezhetőségek felderítése} (angolul \textit{vulnerability assessment}) egy olyan folyamat amely során az infrastruktúrában lévő sebezhetőségeket beazonosítjuk és megállapítjuk a súlyosságaikat kockázatelemzés során.
	
\subsection{Behatolás Tesztelés}
	
	A legoffenzívebb módszere a sebezhetőségek felderítésére a \textit{behatolás tesztelés} (angolul \textit{penetration testing}) amely egy valós támadást szimulál az infrastruktúrán.
	
	\noindent A behatolástesztelést a következő módokon lehet elvégezni:
	
	\begin{itemize}
		\item port a szerveren -- egy bizonyos szolgáltatás tesztelése biztonsági frissítések megléte és helyes beállítások iránt (például egy SMTP szerver);
		\item web alkalmazás -- egy teljes webes alkalmazás bejárása és ismert biztonsági rések tesztelése;
		\item teljes szerver -- az összes port letapogatása és sebezhetőségkeresés indítása a válasz alapján;
		\item teljes hálózat -- egy teljes hálózat letapogatása és tesztelése, például abból a célból, hogy a hálózaton belül az összes szerver amely titkosított információhoz juthat nem sebezhető.
	\end{itemize}
	
	\noindent A behatolástesztelést több megközelítésből is lehet elvégezni:
	
	\begin{itemize}
		\item publikus -- azon támadási vektorok beazomosítására amelyek kihasználhatóak egy kívülálló által (például a publikus IP-címeken lévő szolgáltatások);
		\item kliens -- azon támadási vektorok beazomosítására amelyek kihasználhatóak egy bejelentkezett de nem emelt jogokkal rendelkező felhasználó által (például egy web-banking felhasználó);
		\item belső -- azon károk beazonosítása amelyet egy belső személy képes okozni (például egy rossz-indulatú alkalmazott).
	\end{itemize}

\subsection{Behatolást Megelőző Rendszerek}
	
	A behatolástesztelés egy \textit{aktív} módszere a sebezhetőségek felderítésének, amely egy valós támadást szimulál a hálózaton, viszont ennek a \textit{passzív} megfelelője is létezik \textit{IDS}/\textit{IPS} (\textit{Intrusion Detection/Prevention System}) név alatt.
	
	Ezek a rendszerek a felhasználó és az alkalmazás között helyezkednek el, és a hálózati adatforgalmat figyelik bizonyos lehetséges támadási indikátorokért. Ez a módszer eléggé limitált, ugyanis nem tud olyan sebezhetőségeket észlelni a rendszerben amelyekkel a felhasználó még nem nyúlt, ugyanis az egyetlen adatforrása a rendszernek a valósidejű hálózati adatforgalom.
	
	Egy másik hátránya az IDS típusú rendszereknek, hogy csak figyelnek és nem nyúlnak az adatforgalomhoz, így értesítést küldenek egy támadásról, de azt nem előzik meg, amely bizonyos esetekben azt is jelentheti, hogy a támadó már le is töltötte a bankkártya számokat az adatbázisból.

\section{Kivitelezés}

	A dolgozat céljából fejlesztett alkalmazás és hozzátartozó szkriptek ingyenesek és nyílt forráskódúak.

	A főalkalmazás git repója elérhető a \url{https://github.com/RoliSoft/Host-Scanner} cím alatt, amelyet szabadon fel lehet használni, módosítani illetve terjeszteni a GNU General Public License version 3\cite{gplv3} licencszerződés feltételei alatt.
	
	A különböző hozzátartozó szkipteket, amelyek főként adatfeldolgozási és kísérletezési célokból voltak fejlesztve, a \url{https://github.com/RoliSoft/Host-Scanner-Scripts} címről lehet elérni. Ezeket szabadon fel lehet használni, módosítani illetve terjeszteni az MIT license\cite{mit} licencszerződés feltételei alatt.

\newpage
\section{Bibliográfia}

	\begingroup
	\renewcommand{\section}[2]{}
	\renewcommand{\markboth}[2]{}
		\bibliography{thesis}
		\bibliographystyle{thesis}
	\endgroup

\end{document}